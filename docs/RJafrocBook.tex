\documentclass[]{book}
\usepackage{lmodern}
\usepackage{amssymb,amsmath}
\usepackage{ifxetex,ifluatex}
\usepackage{fixltx2e} % provides \textsubscript
\ifnum 0\ifxetex 1\fi\ifluatex 1\fi=0 % if pdftex
  \usepackage[T1]{fontenc}
  \usepackage[utf8]{inputenc}
\else % if luatex or xelatex
  \ifxetex
    \usepackage{mathspec}
  \else
    \usepackage{fontspec}
  \fi
  \defaultfontfeatures{Ligatures=TeX,Scale=MatchLowercase}
\fi
% use upquote if available, for straight quotes in verbatim environments
\IfFileExists{upquote.sty}{\usepackage{upquote}}{}
% use microtype if available
\IfFileExists{microtype.sty}{%
\usepackage{microtype}
\UseMicrotypeSet[protrusion]{basicmath} % disable protrusion for tt fonts
}{}
\usepackage{hyperref}
\hypersetup{unicode=true,
            pdftitle={RJafroc documentation},
            pdfauthor={Dev P. Chakraborty},
            pdfborder={0 0 0},
            breaklinks=true}
\urlstyle{same}  % don't use monospace font for urls
\usepackage{natbib}
\bibliographystyle{apalike}
\usepackage{color}
\usepackage{fancyvrb}
\newcommand{\VerbBar}{|}
\newcommand{\VERB}{\Verb[commandchars=\\\{\}]}
\DefineVerbatimEnvironment{Highlighting}{Verbatim}{commandchars=\\\{\}}
% Add ',fontsize=\small' for more characters per line
\usepackage{framed}
\definecolor{shadecolor}{RGB}{248,248,248}
\newenvironment{Shaded}{\begin{snugshade}}{\end{snugshade}}
\newcommand{\AlertTok}[1]{\textcolor[rgb]{0.94,0.16,0.16}{#1}}
\newcommand{\AnnotationTok}[1]{\textcolor[rgb]{0.56,0.35,0.01}{\textbf{\textit{#1}}}}
\newcommand{\AttributeTok}[1]{\textcolor[rgb]{0.77,0.63,0.00}{#1}}
\newcommand{\BaseNTok}[1]{\textcolor[rgb]{0.00,0.00,0.81}{#1}}
\newcommand{\BuiltInTok}[1]{#1}
\newcommand{\CharTok}[1]{\textcolor[rgb]{0.31,0.60,0.02}{#1}}
\newcommand{\CommentTok}[1]{\textcolor[rgb]{0.56,0.35,0.01}{\textit{#1}}}
\newcommand{\CommentVarTok}[1]{\textcolor[rgb]{0.56,0.35,0.01}{\textbf{\textit{#1}}}}
\newcommand{\ConstantTok}[1]{\textcolor[rgb]{0.00,0.00,0.00}{#1}}
\newcommand{\ControlFlowTok}[1]{\textcolor[rgb]{0.13,0.29,0.53}{\textbf{#1}}}
\newcommand{\DataTypeTok}[1]{\textcolor[rgb]{0.13,0.29,0.53}{#1}}
\newcommand{\DecValTok}[1]{\textcolor[rgb]{0.00,0.00,0.81}{#1}}
\newcommand{\DocumentationTok}[1]{\textcolor[rgb]{0.56,0.35,0.01}{\textbf{\textit{#1}}}}
\newcommand{\ErrorTok}[1]{\textcolor[rgb]{0.64,0.00,0.00}{\textbf{#1}}}
\newcommand{\ExtensionTok}[1]{#1}
\newcommand{\FloatTok}[1]{\textcolor[rgb]{0.00,0.00,0.81}{#1}}
\newcommand{\FunctionTok}[1]{\textcolor[rgb]{0.00,0.00,0.00}{#1}}
\newcommand{\ImportTok}[1]{#1}
\newcommand{\InformationTok}[1]{\textcolor[rgb]{0.56,0.35,0.01}{\textbf{\textit{#1}}}}
\newcommand{\KeywordTok}[1]{\textcolor[rgb]{0.13,0.29,0.53}{\textbf{#1}}}
\newcommand{\NormalTok}[1]{#1}
\newcommand{\OperatorTok}[1]{\textcolor[rgb]{0.81,0.36,0.00}{\textbf{#1}}}
\newcommand{\OtherTok}[1]{\textcolor[rgb]{0.56,0.35,0.01}{#1}}
\newcommand{\PreprocessorTok}[1]{\textcolor[rgb]{0.56,0.35,0.01}{\textit{#1}}}
\newcommand{\RegionMarkerTok}[1]{#1}
\newcommand{\SpecialCharTok}[1]{\textcolor[rgb]{0.00,0.00,0.00}{#1}}
\newcommand{\SpecialStringTok}[1]{\textcolor[rgb]{0.31,0.60,0.02}{#1}}
\newcommand{\StringTok}[1]{\textcolor[rgb]{0.31,0.60,0.02}{#1}}
\newcommand{\VariableTok}[1]{\textcolor[rgb]{0.00,0.00,0.00}{#1}}
\newcommand{\VerbatimStringTok}[1]{\textcolor[rgb]{0.31,0.60,0.02}{#1}}
\newcommand{\WarningTok}[1]{\textcolor[rgb]{0.56,0.35,0.01}{\textbf{\textit{#1}}}}
\usepackage{longtable,booktabs}
\usepackage{graphicx,grffile}
\makeatletter
\def\maxwidth{\ifdim\Gin@nat@width>\linewidth\linewidth\else\Gin@nat@width\fi}
\def\maxheight{\ifdim\Gin@nat@height>\textheight\textheight\else\Gin@nat@height\fi}
\makeatother
% Scale images if necessary, so that they will not overflow the page
% margins by default, and it is still possible to overwrite the defaults
% using explicit options in \includegraphics[width, height, ...]{}
\setkeys{Gin}{width=\maxwidth,height=\maxheight,keepaspectratio}
\IfFileExists{parskip.sty}{%
\usepackage{parskip}
}{% else
\setlength{\parindent}{0pt}
\setlength{\parskip}{6pt plus 2pt minus 1pt}
}
\setlength{\emergencystretch}{3em}  % prevent overfull lines
\providecommand{\tightlist}{%
  \setlength{\itemsep}{0pt}\setlength{\parskip}{0pt}}
\setcounter{secnumdepth}{5}
% Redefines (sub)paragraphs to behave more like sections
\ifx\paragraph\undefined\else
\let\oldparagraph\paragraph
\renewcommand{\paragraph}[1]{\oldparagraph{#1}\mbox{}}
\fi
\ifx\subparagraph\undefined\else
\let\oldsubparagraph\subparagraph
\renewcommand{\subparagraph}[1]{\oldsubparagraph{#1}\mbox{}}
\fi

%%% Use protect on footnotes to avoid problems with footnotes in titles
\let\rmarkdownfootnote\footnote%
\def\footnote{\protect\rmarkdownfootnote}

%%% Change title format to be more compact
\usepackage{titling}

% Create subtitle command for use in maketitle
\providecommand{\subtitle}[1]{
  \posttitle{
    \begin{center}\large#1\end{center}
    }
}

\setlength{\droptitle}{-2em}

  \title{RJafroc documentation}
    \pretitle{\vspace{\droptitle}\centering\huge}
  \posttitle{\par}
    \author{Dev P. Chakraborty}
    \preauthor{\centering\large\emph}
  \postauthor{\par}
      \predate{\centering\large\emph}
  \postdate{\par}
    \date{2019-08-04}

\usepackage{booktabs}
\usepackage{amsthm}
\makeatletter
\def\thm@space@setup{%
  \thm@preskip=8pt plus 2pt minus 4pt
  \thm@postskip=\thm@preskip
}
\makeatother

\begin{document}
\maketitle

{
\setcounter{tocdepth}{1}
\tableofcontents
}
\hypertarget{prerequisites}{%
\chapter{Prerequisites}\label{prerequisites}}

TBA

\hypertarget{intro}{%
\chapter{Introduction}\label{intro}}

\begin{itemize}
\tightlist
\item
  This is the book desribing the \textbf{RJafroc} packages.
\item
  The name of the book is RJafrocBook
\item
  Modality and treatment are used interchangeably.
\item
  Reader is a generic radiologist, or a computer aided detection algorithm, or any algorithmic ``reader''
\item
  TBA
\end{itemize}

\hypertarget{frocdataformat}{%
\chapter{FROC data format}\label{frocdataformat}}

\hypertarget{introduction}{%
\section{Introduction}\label{introduction}}

\begin{itemize}
\item
  In the free-response ROC (\_\_FROC\_\_) paradigm \citep{RN85} the observer's task is to indicate (i.e., \textbf{mark} the location of) and \textbf{rate} (i.e., assign an ordered label - or confidence level - representing the degree of suspicion) regions in the image that are perceived as suspicious for presence of disease. Accordingly, FROC data consists of \textbf{mark-rating pairs}, where each mark indicates a region \footnote{In order to avoid confusion with the ROI-paradigm, I do not like to use the term ROI to describe the marks made by the observer.} that was considered suspicious for presence of a localized lesion and the rating is the corresponding confidence level. The number of mark-rating pairs on any particular case is a-priori unpredictable. It is a non-negative random integer (i.e., 0, 1, 2, \ldots{}) that depends on the case, the reader and the modality. The relatively unstructured nature of FROC data makes FROC paradigm data more difficult to analyze than ROC paradigm data \footnote{Recall that the ROC paradigm always yields a single rating per case.}.
\item
  By adopting a proximity criterion, each mark is classified by the investigator as a lesion localization (LL) - if it is close to a real lesion - or a non-lesion localization (NL) otherwise.
\item
  The rating can be an integer or quasi- continuous (e.g., 0 -- 100), or a floating point value, as long as higher numbers represent greater confidence in presence of one or more lesions in the ROI \footnote{The directionaliy of the rating is not a limitation. If lower values correspond to increased confidence level, it is only necessary to transform the observed rating by subtracting it from a constant value. The constant value can be chosen arbitrarily, typically as the maximum of all observed ratings, thereby ensuring that the transformed value is always non-negative.}.
\item
  Region-level-normal ratings are stored in the \texttt{NL} field and region-level-abnormal ratings are stored in the \texttt{LL} field.
\end{itemize}

\hypertarget{an-actual-froc-dataset}{%
\section{An actual FROC dataset}\label{an-actual-froc-dataset}}

An actual FROC dataset \citep{RN1882} is included as \texttt{dataset04}, which has the following \texttt{dataset} structure:

\begin{Shaded}
\begin{Highlighting}[]
\KeywordTok{str}\NormalTok{(dataset04)}
\CommentTok{#> List of 8}
\CommentTok{#>  $ NL          : num [1:5, 1:4, 1:200, 1:7] -Inf -Inf 1 -Inf -Inf ...}
\CommentTok{#>  $ LL          : num [1:5, 1:4, 1:100, 1:3] 4 5 4 5 4 3 5 4 4 3 ...}
\CommentTok{#>  $ lesionNum   : int [1:100] 1 1 1 1 1 1 1 1 1 1 ...}
\CommentTok{#>  $ lesionID    : num [1:100, 1:3] 1 1 1 1 1 1 1 1 1 1 ...}
\CommentTok{#>  $ lesionWeight: num [1:100, 1:3] 1 1 1 1 1 1 1 1 1 1 ...}
\CommentTok{#>  $ dataType    : chr "FROC"}
\CommentTok{#>  $ modalityID  : Named chr [1:5] "1" "2" "3" "4" ...}
\CommentTok{#>   ..- attr(*, "names")= chr [1:5] "1" "2" "3" "4" ...}
\CommentTok{#>  $ readerID    : Named chr [1:4] "1" "3" "4" "5"}
\CommentTok{#>   ..- attr(*, "names")= chr [1:4] "1" "3" "4" "5"}
\end{Highlighting}
\end{Shaded}

Examination of the output reveals that:

\begin{itemize}
\tightlist
\item
  The \texttt{dataset} structure is a list with 8 members.
\item
  This is a 5-treatment 4-reader dataset (the lengths of the first and second dimensions, respectively, of the \texttt{NL} and \texttt{LL} arrays). The names of the treatments are in the \texttt{modalityID} array:
\end{itemize}

\begin{Shaded}
\begin{Highlighting}[]
\KeywordTok{attributes}\NormalTok{(dataset04}\OperatorTok{$}\NormalTok{modalityID)}
\CommentTok{#> $names}
\CommentTok{#> [1] "1" "2" "3" "4" "5"}
\end{Highlighting}
\end{Shaded}

For example, the name of the second treatment is \texttt{"2"}.

\begin{itemize}
\tightlist
\item
  The names of the readers are in the \texttt{readerID} array:
\end{itemize}

\begin{Shaded}
\begin{Highlighting}[]
\KeywordTok{attributes}\NormalTok{(dataset04}\OperatorTok{$}\NormalTok{readerID)}
\CommentTok{#> $names}
\CommentTok{#> [1] "1" "3" "4" "5"}
\end{Highlighting}
\end{Shaded}

For example, the name of the second reader is \texttt{"3"}. Apparently reader \texttt{"2"} ``dropped out'' of the study.

\hypertarget{numbers-of-non-diseased-and-diseased-cases}{%
\subsection{Numbers of non-diseased and diseased cases}\label{numbers-of-non-diseased-and-diseased-cases}}

\begin{Shaded}
\begin{Highlighting}[]
\KeywordTok{length}\NormalTok{(dataset04}\OperatorTok{$}\NormalTok{NL[}\DecValTok{1}\NormalTok{,}\DecValTok{1}\NormalTok{,,}\DecValTok{1}\NormalTok{])}
\CommentTok{#> [1] 200}
\KeywordTok{length}\NormalTok{(dataset04}\OperatorTok{$}\NormalTok{LL[}\DecValTok{1}\NormalTok{,}\DecValTok{1}\NormalTok{,,}\DecValTok{1}\NormalTok{])}
\CommentTok{#> [1] 100}
\end{Highlighting}
\end{Shaded}

\begin{itemize}
\item
  The third dimension of the \texttt{NL} array is the total number of \textbf{all} cases, i.e., 200, and the third dimension of the \texttt{LL} array, i.e., 100, is the total number of diseased cases.
\item
  Subtracting the number of diseased cases from the number of all cases yields the number of non-diseased cases.
\item
  Therefore, in this dataset, there are 100 diseased cases and 100 non-diseased cases.
\end{itemize}

\hypertarget{why-dimension-the-nl-array-for-the-total-number-of-cases}{%
\subsection{\texorpdfstring{Why dimension the \texttt{NL} array for the total number of cases?}{Why dimension the NL array for the total number of cases?}}\label{why-dimension-the-nl-array-for-the-total-number-of-cases}}

\begin{itemize}
\tightlist
\item
  Because, in addition to \texttt{LLs}, \texttt{NLs} are possible on diseased cases.
\item
  Only \texttt{LLs} are possible on diseased cases.
\item
  Only \texttt{NLs} are possible on non-diseased cases.
\item
  The missing values are filled in with \texttt{-Inf}.
\end{itemize}

\hypertarget{ratings-on-a-non-diseased-case}{%
\subsection{Ratings on a non-diseased case}\label{ratings-on-a-non-diseased-case}}

\begin{itemize}
\tightlist
\item
  For treatment 1, reader 1 and case 1 (the first non-diseased case), the NL ratings are:
\end{itemize}

\begin{Shaded}
\begin{Highlighting}[]
\NormalTok{dataset04}\OperatorTok{$}\NormalTok{NL[}\DecValTok{1}\NormalTok{,}\DecValTok{1}\NormalTok{,}\DecValTok{1}\NormalTok{,]}
\CommentTok{#> [1] -Inf -Inf -Inf -Inf -Inf -Inf -Inf}
\end{Highlighting}
\end{Shaded}

\hypertarget{the-meaning-of-a-negative-infinity-rating}{%
\subsection{The meaning of a negative infinity rating}\label{the-meaning-of-a-negative-infinity-rating}}

\begin{itemize}
\tightlist
\item
  Obviously, a real rating cannot be negative infinity \footnote{If an observer is so highly confident in the \emph{absence} of a localized lesion, he will simply \emph{not mark} the location in question; if he did, then, logically, he should mark \emph{all} areas in the image that are definitely not lesions; in the FROC paradigm only regions with a reasonable degree of suspicion are marked. The radiologist only wishes to draw attention to regions that are reasonably suspicious; the definition of ``reasonable'' is determined by clinical considerations.}. This value is reserved for \textbf{missing ratings}, and more generally, \textbf{missing marks} \footnote{Since there is a one-to-one correspondence between marks and ratings.}. For example, since all values in the above code chunk are negative infinities, this means this treatment-reader-case combination did not yield any mark-rating pairs. This possibility, alluded to above, is only possible with FROC data. All other paradigms (ROC, LROC and ROI) yield at least one rating per case.
\item
  The length of the fourth dimension of the \texttt{NL} array is determined by that treatment-reader-case combination yielding the maximum number of \texttt{NLs}. Consider the following chunk:
\end{itemize}

\begin{Shaded}
\begin{Highlighting}[]
\ControlFlowTok{for}\NormalTok{ (i }\ControlFlowTok{in} \DecValTok{1}\OperatorTok{:}\DecValTok{5}\NormalTok{) }
  \ControlFlowTok{for}\NormalTok{ (j }\ControlFlowTok{in} \DecValTok{1}\OperatorTok{:}\DecValTok{4}\NormalTok{) }
    \ControlFlowTok{for}\NormalTok{ (k }\ControlFlowTok{in} \DecValTok{1}\OperatorTok{:}\DecValTok{200}\NormalTok{) }
      \ControlFlowTok{if}\NormalTok{ (}\KeywordTok{all}\NormalTok{(dataset04}\OperatorTok{$}\NormalTok{NL[i,j,k,] }\OperatorTok{!=}\StringTok{ }\OperatorTok{-}\OtherTok{Inf}\NormalTok{)) }
        \KeywordTok{cat}\NormalTok{(i, j, k, }\KeywordTok{all}\NormalTok{(dataset04}\OperatorTok{$}\NormalTok{NL[i,j,k,] }\OperatorTok{!=}\StringTok{ }\OperatorTok{-}\OtherTok{Inf}\NormalTok{),}\StringTok{"}\CharTok{\textbackslash{}n}\StringTok{"}\NormalTok{)}
\CommentTok{#> 5 4 192 TRUE}
\end{Highlighting}
\end{Shaded}

\begin{itemize}
\tightlist
\item
  This shows that the fourth dimension of the \texttt{NL} array has to be of length 7 because \emph{one, and only reader}, specifically reader ``4'', made 7 \texttt{NL} marks on a diseased case in treatment ``5''!
\end{itemize}

\hypertarget{ratings-on-a-non-diseased-case-1}{%
\subsection{Ratings on a non-diseased case}\label{ratings-on-a-non-diseased-case-1}}

Unlike non-diseased cases, diseased cases can have both \texttt{NL} and \texttt{LL} ratings.

\begin{itemize}
\tightlist
\item
  For treatment 1, reader 1, case 51 (the 1st diseased case) the NL ratings are:
\end{itemize}

\begin{Shaded}
\begin{Highlighting}[]
\NormalTok{dataset04}\OperatorTok{$}\NormalTok{NL[}\DecValTok{1}\NormalTok{,}\DecValTok{1}\NormalTok{,}\DecValTok{51}\NormalTok{,]}
\CommentTok{#> [1] -Inf -Inf -Inf -Inf -Inf -Inf -Inf}
\NormalTok{dataset04}\OperatorTok{$}\NormalTok{lesionNum[}\DecValTok{1}\NormalTok{]}
\CommentTok{#> [1] 1}
\NormalTok{dataset04}\OperatorTok{$}\NormalTok{LL[}\DecValTok{1}\NormalTok{,}\DecValTok{1}\NormalTok{,}\DecValTok{1}\NormalTok{,]}
\CommentTok{#> [1]    4 -Inf -Inf}
\KeywordTok{mean}\NormalTok{(}\KeywordTok{is.finite}\NormalTok{(dataset04}\OperatorTok{$}\NormalTok{LL))}
\CommentTok{#> [1] 0.3043333}
\end{Highlighting}
\end{Shaded}

. There are only two finite values because this case has two ROI-level-abnormal regions, and 2 plus 2 makes for the assumed 4-regions per case. The corresponding \texttt{\$lesionNum} field is 1.

\begin{Shaded}
\begin{Highlighting}[]
\KeywordTok{mean}\NormalTok{(}\KeywordTok{is.finite}\NormalTok{(dataset04}\OperatorTok{$}\NormalTok{NL[,,}\DecValTok{1}\OperatorTok{:}\DecValTok{50}\NormalTok{,]))}
\CommentTok{#> [1] 0.05942857}
\NormalTok{dataset04}\OperatorTok{$}\NormalTok{NL[}\DecValTok{1}\NormalTok{,}\DecValTok{1}\NormalTok{,}\DecValTok{51}\NormalTok{,]}
\CommentTok{#> [1] -Inf -Inf -Inf -Inf -Inf -Inf -Inf}
\NormalTok{dataset04}\OperatorTok{$}\NormalTok{lesionNum[}\DecValTok{1}\NormalTok{]}
\CommentTok{#> [1] 1}
\NormalTok{dataset04}\OperatorTok{$}\NormalTok{LL[}\DecValTok{1}\NormalTok{,}\DecValTok{1}\NormalTok{,}\DecValTok{1}\NormalTok{,]}
\CommentTok{#> [1]    4 -Inf -Inf}
\KeywordTok{mean}\NormalTok{(}\KeywordTok{is.finite}\NormalTok{(dataset04}\OperatorTok{$}\NormalTok{LL))}
\CommentTok{#> [1] 0.3043333}
\end{Highlighting}
\end{Shaded}

\begin{Shaded}
\begin{Highlighting}[]
\KeywordTok{mean}\NormalTok{(}\KeywordTok{is.finite}\NormalTok{(dataset04}\OperatorTok{$}\NormalTok{NL[,,}\DecValTok{1}\OperatorTok{:}\DecValTok{50}\NormalTok{,]))}
\CommentTok{#> [1] 0.05942857}
\NormalTok{dataset04}\OperatorTok{$}\NormalTok{NL[}\DecValTok{1}\NormalTok{,}\DecValTok{1}\NormalTok{,}\DecValTok{51}\NormalTok{,]}
\CommentTok{#> [1] -Inf -Inf -Inf -Inf -Inf -Inf -Inf}
\NormalTok{dataset04}\OperatorTok{$}\NormalTok{lesionNum[}\DecValTok{1}\NormalTok{]}
\CommentTok{#> [1] 1}
\NormalTok{dataset04}\OperatorTok{$}\NormalTok{LL[}\DecValTok{1}\NormalTok{,}\DecValTok{1}\NormalTok{,}\DecValTok{1}\NormalTok{,]}
\CommentTok{#> [1]    4 -Inf -Inf}
\KeywordTok{mean}\NormalTok{(}\KeywordTok{is.finite}\NormalTok{(dataset04}\OperatorTok{$}\NormalTok{LL))}
\CommentTok{#> [1] 0.3043333}
\end{Highlighting}
\end{Shaded}

\begin{itemize}
\tightlist
\item
  The ratings of the 2 ROI-level-abnormal ROIs on this case are 4. The mean rating over all ROI-level-abnormal ROIs is 3.6785323.
\end{itemize}

\begin{Shaded}
\begin{Highlighting}[]
\KeywordTok{mean}\NormalTok{(}\KeywordTok{is.finite}\NormalTok{(dataset04}\OperatorTok{$}\NormalTok{NL[,,}\DecValTok{1}\OperatorTok{:}\DecValTok{50}\NormalTok{,]))}
\CommentTok{#> [1] 0.05942857}
\NormalTok{dataset04}\OperatorTok{$}\NormalTok{NL[}\DecValTok{1}\NormalTok{,}\DecValTok{1}\NormalTok{,}\DecValTok{51}\NormalTok{,]}
\CommentTok{#> [1] -Inf -Inf -Inf -Inf -Inf -Inf -Inf}
\NormalTok{dataset04}\OperatorTok{$}\NormalTok{lesionNum[}\DecValTok{1}\NormalTok{]}
\CommentTok{#> [1] 1}
\NormalTok{dataset04}\OperatorTok{$}\NormalTok{LL[}\DecValTok{1}\NormalTok{,}\DecValTok{1}\NormalTok{,}\DecValTok{1}\NormalTok{,]}
\CommentTok{#> [1]    4 -Inf -Inf}
\KeywordTok{mean}\NormalTok{(}\KeywordTok{is.finite}\NormalTok{(dataset04}\OperatorTok{$}\NormalTok{LL))}
\CommentTok{#> [1] 0.3043333}
\end{Highlighting}
\end{Shaded}

\hypertarget{the-froc-excel-data-file}{%
\section{The FROC Excel data file}\label{the-froc-excel-data-file}}

An Excel file in JAFROC format containing simulated ROI data corresponding to \texttt{dataset04}, is included with the distribution. The first command (below) finds the location of the file and the second command reads it and saves it to a dataset object \texttt{ds}.

\begin{Shaded}
\begin{Highlighting}[]
\NormalTok{fileName <-}\StringTok{ }\KeywordTok{system.file}\NormalTok{(}
    \StringTok{"extdata"}\NormalTok{, }\StringTok{"includedFrocData.xlsx"}\NormalTok{, }\DataTypeTok{package =} \StringTok{"RJafroc"}\NormalTok{, }\DataTypeTok{mustWork =} \OtherTok{TRUE}\NormalTok{)}
\NormalTok{ds <-}\StringTok{ }\KeywordTok{DfReadDataFile}\NormalTok{(fileName)}
\NormalTok{ds}\OperatorTok{$}\NormalTok{dataType}
\CommentTok{#> [1] "FROC"}
\end{Highlighting}
\end{Shaded}

The \texttt{DfReadDataFile} function automatically recognizes that this is an \emph{ROI} dataset. Its structure is similar to the JAFROC format Excel file, with some important differences, noted below. It contains three worksheets:

\includegraphics[width=0.4\textwidth,height=\textheight]{images/FROC-TP-Truth-1.png}\includegraphics[width=0.4\textwidth,height=\textheight]{images/FROC-TP-Truth-2.png}
\includegraphics[width=0.4\textwidth,height=\textheight]{images/FROC-TP-Truth-3.png}\includegraphics[width=0.4\textwidth,height=\textheight]{images/FROC-TP-Truth-4.png}
\includegraphics[width=0.4\textwidth,height=\textheight]{images/FROC-TP-Truth-5.png}

\begin{itemize}
\tightlist
\item
  The \texttt{Truth} worksheet - this indicates which cases are diseased and which are non-diseased and the number of ROI-level-abnormal region on each case.

  \begin{itemize}
  \tightlist
  \item
    There are 50 normal cases (labeled 1-50) under column \texttt{CaseID} and 40 abnormal cases (labeled 51-90).\\
  \item
    The \texttt{LesionID} field for each normal case (e.g., \texttt{CaseID} = 1) is zero and there is one row per case. For abnormal cases, this field has a variable number of entries, ranging from 1 to 4. As an example, there are two rows for \texttt{CaseID} = 51 in the Excel file: one with \texttt{LesionID} = 2 and one with \texttt{LesionID} = 3.\\
  \item
    The \texttt{Weights} field is always zero (this field is not used in ROI analysis).
  \end{itemize}
\end{itemize}

\includegraphics[width=0.4\textwidth,height=\textheight]{images/ROI-FP-1.png}
\includegraphics[width=0.4\textwidth,height=\textheight]{images/ROI-FP-2.png}

\begin{itemize}
\tightlist
\item
  The \texttt{FP} (or \texttt{NL}) worksheet - this lists the ratings of ROI-level-normal regions.

  \begin{itemize}
  \tightlist
  \item
    For \texttt{ReaderID} = 1, \texttt{ModalityID} = 1 and \texttt{CaseID} = 1 there are 4 rows, corresponding to the 4 ROI-level-normal regions in this case. The corresponding ratings are . The pattern repeats for other treatments and readers, but the rating are, of course, different.\\
  \item
    Each \texttt{CaseID} is represented in the \texttt{FP} worksheet (a rare exception could occur if a case-level abnormal case has 4 abnormal regions).
  \end{itemize}
\end{itemize}

\includegraphics[width=0.4\textwidth,height=\textheight]{images/ROI-TP-1.png}

\begin{itemize}
\tightlist
\item
  The \texttt{TP} (or \texttt{LL}) worksheet - this lists the ratings of ROI-level-abnormal regions.

  \begin{itemize}
  \tightlist
  \item
    Because normal cases generate TPs, one does not find any entry with \texttt{CaseID} = 1-50 in the \texttt{TP} worksheet.\\
  \item
    The lowest \texttt{CaseID} in the \texttt{TP} worksheet is 51, which corresponds to the first abnormal case.\\
  \item
    There are two entries for this case, corresponding to the two ROI-level-abnormal regions present in this case. Recall that corresponding to this \texttt{CaseID} in the \texttt{Truth} worksheet there were two entries with \texttt{LesionID} = 2 and 3. These must match the \texttt{LesionID}'s listed for this case in the \texttt{TP} worksheet. Complementing these two entries, in the \texttt{FP} worksheet for \texttt{CaseID} = 51, there are 2 entries corresponding to the two ROI-level-normal regions in this case.\\
  \item
    One should be satisfied that for each abnormal case the sum of the number of entries in the \texttt{TP} and \texttt{FP} worksheets is always 4.
  \end{itemize}
\end{itemize}

\hypertarget{tba-roi-paradigm}{%
\chapter{TBA ROI paradigm}\label{tba-roi-paradigm}}

\begin{itemize}
\tightlist
\item
  One can think of the ROI paradigm as similar to the FROC paradigm, but with localization accuracy restricted to belonging to a region (one cannot distinguish multiple lesions within a region). The ROIs are defined prior to the study and made known to all observers participating in the study. Unlike the FROC paradigm, a \textbf{rating is required for every ROI}.
\end{itemize}

\hypertarget{references}{%
\section{References}\label{references}}

\bibliography{book.bib,packages.bib,myRefs.bib}


\end{document}
